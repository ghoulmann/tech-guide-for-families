% Generated by Sphinx.
\def\sphinxdocclass{report}
\documentclass[letterpaper,10pt,english]{sphinxmanual}
\usepackage[utf8]{inputenc}
\DeclareUnicodeCharacter{00A0}{\nobreakspace}
\usepackage[T1]{fontenc}
\usepackage{babel}
\usepackage{times}
\usepackage[Bjarne]{fncychap}
\usepackage{longtable}
\usepackage{sphinx}
\usepackage{multirow}


\title{Parent and Guardian Guide to Chelsea School Technology}
\date{November 30, 2013}
\release{13.11}
\author{Rik Goldman}
\newcommand{\sphinxlogo}{}
\renewcommand{\releasename}{Release}
\makeindex

\makeatletter
\def\PYG@reset{\let\PYG@it=\relax \let\PYG@bf=\relax%
    \let\PYG@ul=\relax \let\PYG@tc=\relax%
    \let\PYG@bc=\relax \let\PYG@ff=\relax}
\def\PYG@tok#1{\csname PYG@tok@#1\endcsname}
\def\PYG@toks#1+{\ifx\relax#1\empty\else%
    \PYG@tok{#1}\expandafter\PYG@toks\fi}
\def\PYG@do#1{\PYG@bc{\PYG@tc{\PYG@ul{%
    \PYG@it{\PYG@bf{\PYG@ff{#1}}}}}}}
\def\PYG#1#2{\PYG@reset\PYG@toks#1+\relax+\PYG@do{#2}}

\expandafter\def\csname PYG@tok@gd\endcsname{\def\PYG@tc##1{\textcolor[rgb]{0.63,0.00,0.00}{##1}}}
\expandafter\def\csname PYG@tok@gu\endcsname{\let\PYG@bf=\textbf\def\PYG@tc##1{\textcolor[rgb]{0.50,0.00,0.50}{##1}}}
\expandafter\def\csname PYG@tok@gt\endcsname{\def\PYG@tc##1{\textcolor[rgb]{0.00,0.27,0.87}{##1}}}
\expandafter\def\csname PYG@tok@gs\endcsname{\let\PYG@bf=\textbf}
\expandafter\def\csname PYG@tok@gr\endcsname{\def\PYG@tc##1{\textcolor[rgb]{1.00,0.00,0.00}{##1}}}
\expandafter\def\csname PYG@tok@cm\endcsname{\let\PYG@it=\textit\def\PYG@tc##1{\textcolor[rgb]{0.25,0.50,0.56}{##1}}}
\expandafter\def\csname PYG@tok@vg\endcsname{\def\PYG@tc##1{\textcolor[rgb]{0.73,0.38,0.84}{##1}}}
\expandafter\def\csname PYG@tok@m\endcsname{\def\PYG@tc##1{\textcolor[rgb]{0.13,0.50,0.31}{##1}}}
\expandafter\def\csname PYG@tok@mh\endcsname{\def\PYG@tc##1{\textcolor[rgb]{0.13,0.50,0.31}{##1}}}
\expandafter\def\csname PYG@tok@cs\endcsname{\def\PYG@tc##1{\textcolor[rgb]{0.25,0.50,0.56}{##1}}\def\PYG@bc##1{\setlength{\fboxsep}{0pt}\colorbox[rgb]{1.00,0.94,0.94}{\strut ##1}}}
\expandafter\def\csname PYG@tok@ge\endcsname{\let\PYG@it=\textit}
\expandafter\def\csname PYG@tok@vc\endcsname{\def\PYG@tc##1{\textcolor[rgb]{0.73,0.38,0.84}{##1}}}
\expandafter\def\csname PYG@tok@il\endcsname{\def\PYG@tc##1{\textcolor[rgb]{0.13,0.50,0.31}{##1}}}
\expandafter\def\csname PYG@tok@go\endcsname{\def\PYG@tc##1{\textcolor[rgb]{0.20,0.20,0.20}{##1}}}
\expandafter\def\csname PYG@tok@cp\endcsname{\def\PYG@tc##1{\textcolor[rgb]{0.00,0.44,0.13}{##1}}}
\expandafter\def\csname PYG@tok@gi\endcsname{\def\PYG@tc##1{\textcolor[rgb]{0.00,0.63,0.00}{##1}}}
\expandafter\def\csname PYG@tok@gh\endcsname{\let\PYG@bf=\textbf\def\PYG@tc##1{\textcolor[rgb]{0.00,0.00,0.50}{##1}}}
\expandafter\def\csname PYG@tok@ni\endcsname{\let\PYG@bf=\textbf\def\PYG@tc##1{\textcolor[rgb]{0.84,0.33,0.22}{##1}}}
\expandafter\def\csname PYG@tok@nl\endcsname{\let\PYG@bf=\textbf\def\PYG@tc##1{\textcolor[rgb]{0.00,0.13,0.44}{##1}}}
\expandafter\def\csname PYG@tok@nn\endcsname{\let\PYG@bf=\textbf\def\PYG@tc##1{\textcolor[rgb]{0.05,0.52,0.71}{##1}}}
\expandafter\def\csname PYG@tok@no\endcsname{\def\PYG@tc##1{\textcolor[rgb]{0.38,0.68,0.84}{##1}}}
\expandafter\def\csname PYG@tok@na\endcsname{\def\PYG@tc##1{\textcolor[rgb]{0.25,0.44,0.63}{##1}}}
\expandafter\def\csname PYG@tok@nb\endcsname{\def\PYG@tc##1{\textcolor[rgb]{0.00,0.44,0.13}{##1}}}
\expandafter\def\csname PYG@tok@nc\endcsname{\let\PYG@bf=\textbf\def\PYG@tc##1{\textcolor[rgb]{0.05,0.52,0.71}{##1}}}
\expandafter\def\csname PYG@tok@nd\endcsname{\let\PYG@bf=\textbf\def\PYG@tc##1{\textcolor[rgb]{0.33,0.33,0.33}{##1}}}
\expandafter\def\csname PYG@tok@ne\endcsname{\def\PYG@tc##1{\textcolor[rgb]{0.00,0.44,0.13}{##1}}}
\expandafter\def\csname PYG@tok@nf\endcsname{\def\PYG@tc##1{\textcolor[rgb]{0.02,0.16,0.49}{##1}}}
\expandafter\def\csname PYG@tok@si\endcsname{\let\PYG@it=\textit\def\PYG@tc##1{\textcolor[rgb]{0.44,0.63,0.82}{##1}}}
\expandafter\def\csname PYG@tok@s2\endcsname{\def\PYG@tc##1{\textcolor[rgb]{0.25,0.44,0.63}{##1}}}
\expandafter\def\csname PYG@tok@vi\endcsname{\def\PYG@tc##1{\textcolor[rgb]{0.73,0.38,0.84}{##1}}}
\expandafter\def\csname PYG@tok@nt\endcsname{\let\PYG@bf=\textbf\def\PYG@tc##1{\textcolor[rgb]{0.02,0.16,0.45}{##1}}}
\expandafter\def\csname PYG@tok@nv\endcsname{\def\PYG@tc##1{\textcolor[rgb]{0.73,0.38,0.84}{##1}}}
\expandafter\def\csname PYG@tok@s1\endcsname{\def\PYG@tc##1{\textcolor[rgb]{0.25,0.44,0.63}{##1}}}
\expandafter\def\csname PYG@tok@gp\endcsname{\let\PYG@bf=\textbf\def\PYG@tc##1{\textcolor[rgb]{0.78,0.36,0.04}{##1}}}
\expandafter\def\csname PYG@tok@sh\endcsname{\def\PYG@tc##1{\textcolor[rgb]{0.25,0.44,0.63}{##1}}}
\expandafter\def\csname PYG@tok@ow\endcsname{\let\PYG@bf=\textbf\def\PYG@tc##1{\textcolor[rgb]{0.00,0.44,0.13}{##1}}}
\expandafter\def\csname PYG@tok@sx\endcsname{\def\PYG@tc##1{\textcolor[rgb]{0.78,0.36,0.04}{##1}}}
\expandafter\def\csname PYG@tok@bp\endcsname{\def\PYG@tc##1{\textcolor[rgb]{0.00,0.44,0.13}{##1}}}
\expandafter\def\csname PYG@tok@c1\endcsname{\let\PYG@it=\textit\def\PYG@tc##1{\textcolor[rgb]{0.25,0.50,0.56}{##1}}}
\expandafter\def\csname PYG@tok@kc\endcsname{\let\PYG@bf=\textbf\def\PYG@tc##1{\textcolor[rgb]{0.00,0.44,0.13}{##1}}}
\expandafter\def\csname PYG@tok@c\endcsname{\let\PYG@it=\textit\def\PYG@tc##1{\textcolor[rgb]{0.25,0.50,0.56}{##1}}}
\expandafter\def\csname PYG@tok@mf\endcsname{\def\PYG@tc##1{\textcolor[rgb]{0.13,0.50,0.31}{##1}}}
\expandafter\def\csname PYG@tok@err\endcsname{\def\PYG@bc##1{\setlength{\fboxsep}{0pt}\fcolorbox[rgb]{1.00,0.00,0.00}{1,1,1}{\strut ##1}}}
\expandafter\def\csname PYG@tok@kd\endcsname{\let\PYG@bf=\textbf\def\PYG@tc##1{\textcolor[rgb]{0.00,0.44,0.13}{##1}}}
\expandafter\def\csname PYG@tok@ss\endcsname{\def\PYG@tc##1{\textcolor[rgb]{0.32,0.47,0.09}{##1}}}
\expandafter\def\csname PYG@tok@sr\endcsname{\def\PYG@tc##1{\textcolor[rgb]{0.14,0.33,0.53}{##1}}}
\expandafter\def\csname PYG@tok@mo\endcsname{\def\PYG@tc##1{\textcolor[rgb]{0.13,0.50,0.31}{##1}}}
\expandafter\def\csname PYG@tok@mi\endcsname{\def\PYG@tc##1{\textcolor[rgb]{0.13,0.50,0.31}{##1}}}
\expandafter\def\csname PYG@tok@kn\endcsname{\let\PYG@bf=\textbf\def\PYG@tc##1{\textcolor[rgb]{0.00,0.44,0.13}{##1}}}
\expandafter\def\csname PYG@tok@o\endcsname{\def\PYG@tc##1{\textcolor[rgb]{0.40,0.40,0.40}{##1}}}
\expandafter\def\csname PYG@tok@kr\endcsname{\let\PYG@bf=\textbf\def\PYG@tc##1{\textcolor[rgb]{0.00,0.44,0.13}{##1}}}
\expandafter\def\csname PYG@tok@s\endcsname{\def\PYG@tc##1{\textcolor[rgb]{0.25,0.44,0.63}{##1}}}
\expandafter\def\csname PYG@tok@kp\endcsname{\def\PYG@tc##1{\textcolor[rgb]{0.00,0.44,0.13}{##1}}}
\expandafter\def\csname PYG@tok@w\endcsname{\def\PYG@tc##1{\textcolor[rgb]{0.73,0.73,0.73}{##1}}}
\expandafter\def\csname PYG@tok@kt\endcsname{\def\PYG@tc##1{\textcolor[rgb]{0.56,0.13,0.00}{##1}}}
\expandafter\def\csname PYG@tok@sc\endcsname{\def\PYG@tc##1{\textcolor[rgb]{0.25,0.44,0.63}{##1}}}
\expandafter\def\csname PYG@tok@sb\endcsname{\def\PYG@tc##1{\textcolor[rgb]{0.25,0.44,0.63}{##1}}}
\expandafter\def\csname PYG@tok@k\endcsname{\let\PYG@bf=\textbf\def\PYG@tc##1{\textcolor[rgb]{0.00,0.44,0.13}{##1}}}
\expandafter\def\csname PYG@tok@se\endcsname{\let\PYG@bf=\textbf\def\PYG@tc##1{\textcolor[rgb]{0.25,0.44,0.63}{##1}}}
\expandafter\def\csname PYG@tok@sd\endcsname{\let\PYG@it=\textit\def\PYG@tc##1{\textcolor[rgb]{0.25,0.44,0.63}{##1}}}

\def\PYGZbs{\char`\\}
\def\PYGZus{\char`\_}
\def\PYGZob{\char`\{}
\def\PYGZcb{\char`\}}
\def\PYGZca{\char`\^}
\def\PYGZam{\char`\&}
\def\PYGZlt{\char`\<}
\def\PYGZgt{\char`\>}
\def\PYGZsh{\char`\#}
\def\PYGZpc{\char`\%}
\def\PYGZdl{\char`\$}
\def\PYGZhy{\char`\-}
\def\PYGZsq{\char`\'}
\def\PYGZdq{\char`\"}
\def\PYGZti{\char`\~}
% for compatibility with earlier versions
\def\PYGZat{@}
\def\PYGZlb{[}
\def\PYGZrb{]}
\makeatother

\begin{document}

\maketitle
\tableofcontents
\phantomsection\label{Index::doc}


Contents:


\chapter{Moodle}
\label{moodle:chelsea-school-technology-for-parents}\label{moodle::doc}\label{moodle:moodle}

\section{What is a Moodle?}
\label{moodle:what-is-a-moodle}
\href{https://moodle.org/about/}{Moodle} is a ``learning management system,'' a phrase used to refer to a class of software  ``for the administration, documentation, tracking, reporting and delivery of e-learning education courses or training programs.'' \footnote{
Ellis, Ryann K. (2009), Field Guide to Learning Management Systems, ASTD Learning Circuits.
}

Institutions rely on Moodle to fulfill different functions. At \href{http://chelseaschool.edu}{Chelsea School}, we rely on Moodle to augment on-campus courses or to provide hybrid courses.

Moodle is accessible via the Web at the following address: \href{http://chelseapride.org}{http://chelseapride.org} - note the address does not begin with \emph{www}.

Chelsea School's Moodle works best with Google Chrome, Mozilla Firefox, Opera, and Safari.


\section{Front Page (Main Conent Area)}
\label{moodle:front-page-main-conent-area}
We've made a significant effort to make Moodle's front page more accessible to more students.

In the past, students have been faced with a wall of text listing courses and course catalogs. Beginning this year, the main part of the Front Page begins with a more visual experience; the verbal components are still on the front page, but are now subordinate to the icon-driven experience.

Icons appear in three groups: Quicklinks for students, quicklinks for parents and guardians, and finally course categories.


\subsection{Links for Students}
\label{moodle:links-for-students}
In addition to a quicklink to course icons, students will find links for accessing other resources, such as their school E-mail accounts, Khan Academy, etc.


\subsection{Links for Parents and Guardians}
\label{moodle:links-for-parents-and-guardians}
This section includes links to PowerSchool, the student handbook, etc.


\subsection{Academic Departments}
\label{moodle:academic-departments}
We've limited the amount of text the user has to cut through by linking directly to course departments: Science, Technology, English/Language Arts, etc.


\subsection{Courses by Category}
\label{moodle:courses-by-category}
Wall of text. More direct path to courses, but not the best way to serve many in our community.


\section{Front Page (Left Sidebar)}
\label{moodle:front-page-left-sidebar}

\subsection{Logging In (Students Only)}
\label{moodle:logging-in-students-only}
To the immediate left of the front page content area is a sidebar for navigation. It is divided into blocks. The top block is for logging in to Moodle.

Logging in requires a username and complex password. Both are provided by a student's advisor and cannot be changed by the student. The pattern for student usernames is \textless{}lastname\textgreater{}\textless{}first initial\textgreater{} (all lower case). To log in, enter the username in the username field and the password into the password field; press enter or click \emph{Login} to submit credentials.

After a brief wait, successful login will be indicated at the top right of the page, which should indicate that the student is the logged in user.

Forgotten passwords can be recovered by advisors and their colleagues. However, students who get stuck logging in to Moodle from home can contact \href{http://chelseapride.org/helpdesk}{helpdesk} at \href{http://chelseapride.org/helpdesk}{http://chelseapride.org/helpdesk}.

There are icons for helpdesk in both the Student's section of the front page and the Parents' and Guardians' section of the front page.


\section{Parent \& Guardian Access}
\label{moodle:parent-guardian-access}
Parents and guardians are discouraged from logging in to student accounts; we understand, however, that in some cases it can be helpful.

With that in mind, we've configured all available courses to be either:
\begin{enumerate}
\item {} 
Available to the public; or

\item {} 
Available with a passcode.

\end{enumerate}

At Chelsea School, restricted courses have been configured to provide guest access to guests with a course passcode.

The passcode with which parents and guardians can access all restricted courses is \textbf{chelseaguest}.

To access a course as a guest,
\begin{enumerate}
\item {} 
Navigate to the desired course;

\item {} 
Click on ``Login as Guest;''

\item {} 
Enter ``Guest Access Password'' and click ``Submit.''

\end{enumerate}


\section{Navigating to a Course}
\label{moodle:navigating-to-a-course}
Moodle provides several ways to access courses:
\begin{enumerate}
\item {} 
On the left sidebar of the front page, there's a ``block'' called ``Navigation.'' Select ``courses'' in the Navigation block to expand to show course categories; to view individual courses within a category, select the category; the contents will expand to show available courses with abbreviated names. Click on the name of the course to access it.

\item {} 
In the ``Student Stuff'' section of the front page (main content area), there's an icon labeled ``courses.'' Click on the icon to see icons for each department in the main content area of the front page. Click on the department of the desired course to reveal a list of the department's courses in the main content area. Browse for the name of the desired course and click on it.

\item {} 
From the front page, one can search for courses by scrolling down to below the course categories section in the main content area. To search for a course, accurately enter a portion of the course name and press enter (or click the blue ``Go'' button.

\item {} 
To browse the full course catalog by department, scroll down to just past the search field on the front page (main content area). Click on the desired course to access it.

\end{enumerate}


\section{Resources}
\label{moodle:helpdesk}\label{moodle:resources}\begin{itemize}
\item {} 
\href{http://youtu.be/PSwbhCqLSQI}{Introducing Parents to Moodle} (YouTube)

\item {} 
\href{http://docs.moodle.org/26/en/Moodle\_site\_-\_basic\_structure}{Moodle Site Structure} (Documentation)

\item {} 
\href{http://learning.oconeeschools.org/course/view.php?id=477}{Moode for Students and Parents} (Frequently Asked Questions)

\end{itemize}

\index{moodle}\index{LMS}\index{CMS}\index{courses}\index{departments}\index{key}\index{passcode}\index{password}\index{login}\index{logging in}\index{credentials}\index{icons}\index{visual}

\chapter{PowerSchool}
\label{powerschool::doc}\label{powerschool:powerschool}\label{powerschool:index-0}

\section{What is PowerSchool?}
\label{powerschool:what-is-powerschool}
Pearson's \href{http://chelseaschool.powerschool.com/}{PowerSchool} is a web-based shool information system. At its foundation, PowerSchool is used to record, report, and view student grades, attendance, and reports.


\section{Credentials}
\label{powerschool:credentials}
Credentials for \href{http://chelseaschool.powerschool.com/}{PowerSchool} Parent Portal were generated for each student's parents and guardians and sent via E-mail at the beginning of Quarter 1. If you've misplaced your login credentials, please contact \href{http://chelseapride.org/helpdesk}{helpdesk}.


\section{Accessing PowerSchool}
\label{powerschool:accessing-powerschool}
It's been our experience that \href{http://chelseaschool.powerschool.com/}{Powerschool} is best accessed using Mozilla \href{http://www.mozilla.org/en-US/firefox/new/}{Firefox}. We have compelling evidence that Google Chrome is not compatible at this point.

Using \href{http://www.mozilla.org/en-US/firefox/new/}{Firefox}, navigate to \href{http://chelseaschool.powerschool.com/}{http://chelseaschool.powerschool.com/}.

Alternatively, navigate to \href{http://chelseapride.org}{Moodle} (\href{http://chelseapride.org}{http://chelseapride.org}) and click the \href{http://chelseaschool.powerschool.com/}{PowerSchool} icon on the front page under either ``Student Stuff'' or ``Family Stuff.''

You will prompted to enter credentials for PowerSchool. Enter the correct credentials, being mindful of capitalization, and click the ``Sign In'' button. (From this page, you may also follow links to recover a lost username or password.)

If you have trouble accessing \href{http://chelseaschool.powerschool.com/}{PowerSchool} the first time, please contact \href{http://chelseapride.org/helpdesk}{helpdesk} via the Web. Please include the following information: your name, the student's name, and a description of the problem your having in as much detail as you can provide.


\section{Resources}
\label{powerschool:resources}\begin{itemize}
\item {} 
\href{http://youtu.be/SzF4wF4fglY}{PowerSchool Parent Portal} (YouTube)

\item {} 
\href{http://youtu.be/7z5rOk-89OE}{PowerSchool: Parent Portal Tutorial} (YouTube)

\item {} 
\href{http://youtu.be/YFS0n2D8ri4}{Viewing Child's Grades in PowerSchool Parent Portal} (YouTube)

\item {} 
Parent Portal User Guide (PDF)

\end{itemize}

\index{grades}\index{attendance}\index{transcripts}\index{reports}\index{students}\index{PowerSchool}\index{Pearson}\index{SIS}\index{parents}\index{guardians}

\chapter{Indices and tables}
\label{Index:indices-and-tables}\label{Index:index-0}\begin{itemize}
\item {} 
\emph{genindex}

\item {} 
\emph{search}

\end{itemize}



\renewcommand{\indexname}{Index}
\printindex
\end{document}
